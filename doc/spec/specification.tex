\documentclass[10pt, a4paper]{article}
\newcommand{\Rectangle}[3]
{
  \realmult{#1}{0.5}\resultx
  \realmult{#2}{0.5}\resulty
  \rmove({\resultx} 0)
  \bsegment
    \move({-\resultx} {\resulty})
    \lvec({\resultx} {\resulty})
    \lvec({\resultx} {-\resulty})
    \lvec({-\resultx} {-\resulty})
    \lvec({-\resultx} {\resulty})
    \textref h:C v:C
    \htext(0 0){\hbox{#3}}
    \savepos({\resultx} 0)(*ex *ey)
  \esegment
  \move(*ex *ey)
}
\newcommand{\CylindreV}[4]
{
  \rmove(#1 0)
  \bsegment
    \move(0 #3)
    \lellip rx:#1 ry:#2
    \move(-#1 #3)
    \lvec(-#1 -#3)
    \move(0 -#3)
    \lellip rx:#1 ry:#2
    \move(#1 #3)
    \lvec(#1 -#3)
    \textref h:C v:C
    \htext(0 0){\vbox{#4}}
    \savepos(#1 0)(*ex *ey)
  \esegment
  \move(*ex *ey)
}
\newcommand{\CylindreH}[4]
{
  \realmult{#3}{0.5}\result
  \rmove({\result} 0)
  \bsegment
    \move({-\result} 0)
    \lellip rx:#1 ry:#2
    \move({-\result} -#2)
    \lvec({\result} -#2)
    \move({\result} 0)
    \lellip rx:#1 ry:#2
    \move({-\result} #2)
    \lvec({\result} #2)
    \textref h:C v:C
    \htext(0 0){\hbox{#4}}
    \savepos({\result} 0)(*ex *ey)
  \esegment
  \move(*ex *ey)
}
\newcommand{\Cercle}[2]
{
  \rmove(#1 0)
  \bsegment
    \lcir r:#1
    \textref h:C v:C
    \htext(0 0){\hbox{#2}}
    \savepos(#1 0)(*ex *ey)
  \esegment
  \move(*ex *ey)
}
\newcommand{\Vecteur}[3]
{
  \realmult{#1}{0.5}\result
  \bsegment
    \ravec(#1 #2)
    \textref h:C v:C
    \htext({\result} 0.1){\hbox{#3}}
    \savepos(#1 #2)(*ex *ey)
  \esegment
  \move(*ex *ey)
}
\newcommand{\LigneV}[3]
{
  \realmult{#2}{0.5}\result
  \bsegment
    \rlvec(#1 #2)
    \textref h:C v:C
    \htext(0.1 {\result}){\hbox{#3}}
    \savepos(#1 #2)(*ex *ey)
  \esegment
  \move(*ex *ey)
}


\usepackage[utf8]{inputenc}
\usepackage[LAE]{fontenc}
\usepackage[english, arabic]{babel}
%\usepackage[usenames]{color}
%\usepackage[colorlinks,backref]{hyperref}
\usepackage{texdraw}
\begin{document}
\selectlanguage{english}
\title{{\bf D}iscreete {\bf T}ime {\bf S}imulator {\bf L}ibraries\\Specification Document}
\author{Mazen Neifer}
%\date{June 2007}
\maketitle
\newpage
\selectlanguage{arabic}
\begin{center}
{\huge بسم الله الرحمان الرحيم}
\end{center}
\section{مقدمة}
بحمد الله نقدم فيما يلي عرضا لإشكالية صنع برامج المحاكاة \textLR{(simultion programs)} و التي تعتمد على مجموعة من الوسائل الجاهزة
\section{طرح الإشكالية}
\begin{otherlanguage}{english}
\par The main problem duscussed in this document is the creation of a set of tools easing creation of a simulator. Plese note that any system can be modeled as following
\begin{figure}[h]
\begin{texdraw}
  \Vecteur{0.5}{0.0}{$\vec X$}
  \CylindreH{0.2}{0.3}{1.0}{FIFO($\vec X$)}
  \Vecteur{0.5}{0.0}{$\vec X$}
  \Rectangle{0.8}{0.6}{$\vec Y=f(\vec X)$}
  \Vecteur{0.5}{0.0}{$\vec Y$}
  \CylindreH{0.2}{0.3}{1.0}{FIFO($\vec Y$)}
  \Vecteur{0.5}{0.0}{$\vec Y$}
\end{texdraw}
\caption{example of draw}
\end{figure}
\end{otherlanguage}
\par
تم والحمد لله رب العالمين
\end{document}
